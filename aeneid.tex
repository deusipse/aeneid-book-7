\documentclass[11pt, a4paper, landscape]{article}
\usepackage{fontspec}
\usepackage{geometry}
\geometry{
  top = 0.5in,
  bottom = 0.75in,
  left = 2in,
  right = 1.25in
}
\usepackage[series={}, nocritical, noeledsec, noend, nofamiliar, noledgroup]{reledmac}
\setlinenum{322}
\usepackage{reledpar}
\usepackage{microtype}
\usepackage{setspace}
% \usepackage{newtxtext}

\setmainfont{Times New Roman}
\setsansfont{Arial}

\usepackage{lipsum}
\usepackage{multicol}
\setlength\columnsep{1.75in}
\usepackage{parskip}
\usepackage[pdfusetitle, hidelinks]{hyperref}
\usepackage[english, main=latin]{babel}
\babeltags{english = english}

\newcommand{\Translation}[3]{
  \textbf{\large \sffamily #1}
  \begin{multicols}{2}
    \begin{spacing}{1.7}
      \beginnumbering
      \pstart
      #2
      \pend
      \pausenumbering
    \end{spacing}
    \columnbreak
    #3
  \end{multicols}
}
\newcommand{\translation}[3]{
  \textbf{\large \sffamily #1}
  \begin{multicols}{2}
    \begin{spacing}{1.7}
      \resumenumbering
      \pstart
      #2
      \pend
      \pausenumbering
    \end{spacing}
    \columnbreak
    #3
  \end{multicols}
}


\begin{document}

\Translation{323--340: Juno Summons Allecto}{
  haec ubi dicta dedit, terras horrenda petivit;\\
  luctificam Allecto dirarum ab sede dearum\\
  infernisque ciet tenebris, cui tristia bella\\
  iraeque insidiaeque et crimina noxia cordi.\\
  odit et ipse pater Pluton, odere sorores\\
  Tartareae monstrum: tot sese vertit in ora,\\
  tam saevae facies, tot pullulat atra colubris.\\
  quam Iuno his acuit verbis ac talia fatur:\\
  `hunc mihi da proprium, virgo sata Nocte, laborem,\\
  hanc operam, ne noster honos infractave cedat\\
  fama loco, neu conubiis ambire Latinum\\
  Aeneadae possint Italosve obsidere finis.\\
  tu potes unanimos armare in proelia fratres\\
  atque odiis versare domos, tu verbera tectis\\
  funereasque inferre faces, tibi nomina mille,\\
  mille nocendi artes. fecundum concute pectus,\\
  dissice compositam pacem, sere crimina belli;\\
  arma velit poscatque simul rapiatque iuventus.'
}{
  When she had spoken these words, the dreadful one sought the earth; she summoned grief bringing Allecto from the seat of the fearful Furies in the infernal darkness, dear to her heart are gloomy wars, anger, trickery, and injurious crimes. And her own father Pluto himself hates her, her sisters of Tartarus hate the monster: she changes herself into so many appearances, such savage forms, she sprouts so many black snakes. Juno stirred her up with these words and spoke such things:

  `Do me this service, virgin daughter of the Night, this task, so that our honour and reputation are not weakened and give way, and see to it that people of Aeneas are unable to win over Latinus with marriage and besiege the Italian lands. You can take brothers who love each other and make them take up arms against themselves in battle, and you can upturn homes with hatred and fill them with whips and funeral torches, you have a thousand names, a thousand ways of causing hurt. Shake your teeming chest, shatter the peace they have put together, sow the accusations of war; let the youth long for weapons, demand them, seize them, now!'
}
\newpage

\translation{341--358: Allecto Attacks Amata}{
  exim Gorgoneis Allecto infecta venenis\\
  principio Latium et Laurentis tecta tyranni\\
  celsa petit, tacitumque obsedit limen Amatae,\\
  quam super adventu Teucrum Turnique hymenaeis\\
  femineae ardentem curaeque iraeque coquebant.\\
  huic dea caeruleis unum de crinibus anguem\\
  conicit, inque sinum praecordia ad intima subdit,\\
  quo furibunda domum monstro permisceat omnem.\\
  ille inter vestis et levia pectora lapsus\\
  volvitur attactu nullo, fallitque furentem\\
  vipeream inspirans animam; fit tortile collo\\
  aurum ingens coluber, fit longae taenia vittae\\
  innectitque comas et membris lubricus errat.\\
  ac dum prima lues udo sublapsa veneno\\
  pertemptat sensus atque ossibus implicat ignem\\
  necdum animus toto percepit pectore flammam,\\
  mollius et solito matrum de more locuta est,\\
  multa super natae lacrimans Phrygiisque hymenaeis:
}{
  At once Allecto, steeped in the poisons of the Gorgons, first sought Latium and the lofty roofs of the Laurentine king, and took up position at the quiet threshold of Amata, who was being stirred up by womanly concerns and anger and was seething at the arrival of Trojans and at the wedding of Turnus. The goddess hurled one of the snakes from her dark blue hair at her, and plunged it into her breast deep into her heart, so that maddened by the monster, she would throw into disarray the whole house. It slid between her robes and smooth breasts, it coiled without her any notice, and without her knowing it, breathed its viperous breath into the frenzied one; the giant snake became a necklace of twisted gold around her neck, it became the hanging end of a long ribbon and fastened itself into her hair, slithering along her body. And while the first infection from the clammy poison was agitating her senses and entwining fire in her bones, before her mind perceived the flame in her whole chest, she spoke softly and in the usual manner of mothers, weeping greatly over the marriage of her daughter to the Phrygians.
}
\newpage

\translation{359--377: Amata Laments Lavinia's Marriage}{
  `exsulibusne datur ducenda Lavinia Teucris,\\
  o genitor, nec te miseret nataeque tuique?\\
  nec matris miseret, quam primo Aquilone relinquet\\
  perfidus alta petens abducta virgine praedo?\\
  at non sic Phrygius penetrat Lacedaemona pastor,\\
  Ledaeamque Helenam Troianas vexit ad urbes?\\
  quid tua sancta fides? quid cura antiqua tuorum\\
  et consanguineo totiens data dextera Turno?\\
  si gener externa petitur de gente Latinis,\\
  idque sedet, Faunique premunt te iussa parentis,\\
  omnem equidem sceptris terram quae libera nostris\\
  dissidet externam reor et sic dicere divos.\\
  et Turno, si prima domus repetatur origo,\\
  Inachus Acrisiusque patres mediaeque Mycenae.'\\
  his ubi nequiquam dictis experta Latinum\\
  contra stare videt, penitusque in viscera lapsum\\
  serpentis furiale malum totamque pererrat,\\
  tum vero infelix ingentibus excita monstris\\
  immensam sine more furit lymphata per urbem.
}{
  `Must Lavinia be led and given in marriage to Trojan exiles, O father, do you have no pity for your daughter or for yourself? Nor do you pity her mother, whom the treacherous pirate will abandon, seeking the high seas as soon as the North wind blows, taking away our virgin? But is this not how the Phrygian shepherd made his way into Lacedaemon, and dragged Helen, the daughter of Leda, off to the cities of Troy? What of your sacred pledge? What of your ancient care for your people, and your right hand so often given to your kinsman Turnus? If a son-in-law is being sought from a people foreign to the Latins, and that is decided upon, if the commands of your father Faunus weigh upon you, then I personally think that all land that is free and separate from our rule is foreign, and such is what the gods say. And if the first origin of the house of Turnus was traced back, his fathers were Inachus and Acrisius and the middle of Mycenae.'

  When having tried in vain with these words she saw that Latinus was standing against her, and the snake's maddening venom had soaked into her flesh and was coursing through her whole body, then indeed the unlucky queen, driven by the monstrous horrors, raged frantically from end to end of the city without regard for convention.
}
\newpage

\translation{378--396: Amata Rages Through the City}{
  ceu quondam torto volitans sub verbere turbo,\\
  quem pueri magno in gyro vacua atria circum\\
  intenti ludo exercent---ille actus habena\\
  curvatis fertur spatiis; stupet inscia supra\\
  impubesque manus mirata volubile buxum;\\
  dant animos plagae: non cursu segnior illo\\
  per medias urbes agitur populosque ferocis.\\
  quin etiam in silvas simulato numine Bacchi\\
  maius adorta nefas maioremque orsa furorem\\
  evolat et natam frondosis montibus abdit,\\
  quo thalamum eripiat Teucris taedasque moretur,\\
  euhoe Bacche fremens, solum te virgine dignum\\
  vociferans: etenim mollis tibi sumere thyrsos,\\
  te lustrare choro, sacrum tibi pascere crinem.\\
  fama volat, furiisque accensas pectore matres\\
  idem omnis simul ardor agit nova quaerere tecta.\\
  deseruere domos, ventis dant colla comasque;\\
  ast aliae tremulis ululatibus aethera complent\\
  pampineasque gerunt incinctae pellibus hastas.
}{
  Just like sometimes a spinning top, flying under the whirled whip, which boys, engrossed in their game, make go in a great circle around an empty court---driven on by the whip it speeds in circular courses; the childish group hang over it in wonder, mesmerised by what they have never seen before, gazing at the twirling boxwood; their strokes give it life: no slower than the spinning top she is driven through the midst of the cities and proud peoples. And moreover she flew into the forests feigning the spirit of Bacchus, she rose to a greater evil and greater madness, and she hid her daughter in the leafy mountains hoping to snatch the marriage away from the Trojans and delay the marriage torches, `euhoe, Bacchus!' she screams, `you alone are worthy of the virgin! For you in truth she lifts the soft thyrsus and moves around you in ritual dance, she grows her sacred hair for you.' Rumour flew, and the same flame in their breasts drove all the mothers, inflamed with frenzy, to seek new homes. They abandoned their houses, baring their neck and hair to the winds; some filled the heavens with tremulous wailing and clad in animal skins, took up vine-draped spears.
}
\newpage

\translation{397--414: Allecto Seeks Out Turnus}{
  ipsa inter medias flagrantem fervida pinum\\
  sustinet ac natae Turnique canit hymenaeos\\
  sanguineam torquens aciem, torvumque repente\\
  clamat: `io matres, audite, ubi quaeque, Latinae:\\
  si qua piis animis manet infelicis Amatae\\
  gratia, si iuris materni cura remordet,\\
  solvite crinalis vittas, capite orgia mecum.'\\
  talem inter silvas, inter deserta ferarum\\
  reginam Allecto stimulis agit undique Bacchi.\\
  postquam visa satis primos acuisse furores\\
  consiliumque omnemque domum vertisse Latini,\\
  protinus hinc fuscis tristis dea tollitur alis\\
  audacis Rutuli ad muros, quam dicitur urbem\\
  Acrisioneis Danae fundasse colonis\\
  praecipiti delata Noto. locus Ardea quondam\\
  dictus avis, et nunc magnum manet Ardea nomen,\\
  sed fortuna fuit. tectis hic Turnus in altis\\
  iam mediam nigra carpebat nocte quietem.
}{
  The fiery queen herself held up a burning pine and sang wedding hymns for Turnus and her daughter, rolling her bloodshot eyes, suddenly she gave a wild shout: 

`Io, mothers of Latium, listen, wherever you are: if any regard for unlucky Amata remains in your faithful hearts, if concern for a mother's rights bites at you, loosen your headbands and join me in the secret rites!'

Such was the queen whom Allecto drove far and wide through the woods and the deserted lairs of wild beasts with the lash of Bacchus. After she saw that she had stirred the first frenzies enough and had subverted the plans and the whole household of Latinus, the gloomy goddess was carried on her dark wings straight to the walls of the bold Rutilian, the city which Danae is said to have founded for the Acrisionean settlers, having been carried there by the swift South wind. The place was once called Ardea by our ancestors, and now the great name Ardea remains, but its greatness is gone. Here in his high palace in the dark of night, Turnus was lying in the middle of his sleep.
}
\newpage

\translation{415--434: Allecto Rouses Turnus}{
  Allecto torvam faciem et furialia membra\\
  exuit, in vultus sese transformat anilis\\
  et frontem obscenam rugis arat, induit albos\\
  cum vitta crinis, tum ramum innectit olivae;\\
  fit Calybe Iunonis anus templique sacerdos,\\
  et iuveni ante oculos his se cum vocibus offert:\\
  `Turne, tot incassum fusos patiere labores,\\
  et tua Dardaniis transcribi sceptra colonis?\\
  rex tibi coniugium et quaesitas sanguine dotes\\
  abnegat, externusque in regnum quaeritur heres.\\
  i nunc, ingratis offer te, inrise, periclis;\\
  Tyrrhenas, i, sterne acies, tege pace Latinos.\\
  haec adeo tibi me, placida cum nocte iaceres,\\
  ipsa palam fari omnipotens Saturnia iussit.\\
  quare age et armari pubem portisque moveri\\
  laetus in arva para, et Phrygios qui flumine pulchro\\
  consedere duces pictasque exure carinas.\\
  caelestum vis magna iubet. rex ipse Latinus,\\
  ni dare coniugium et dicto parere fatetur,\\
  sentiat et tandem Turnum experiatur in armis.'
}{
  Allecto shook off her savage appearance and her fury's writhing limbs, she transformed herself into the form of an old woman and furrowed her brow with repulsive wrinkles, she put on white hair with a fillet, then fastened an olive wreath there; she became Calybe, the aged priestess of Juno, and before the eyes of the young man she offered herself with these words:

`Turnus, will you allow so many efforts of yours to be wasted in vain, and your sceptre handed over to Dardanian colonists? The king denies you your marriage and dowry earned in blood, and a stranger is being sought as heir to the kingdom. Go now, offer yourself, laughed at, to those unthankful dangers; go and cut down the Tyrrhenian battle lines, protect the Latins with peace! The all-powerful Saturnian goddess herself commanded me to openly give these messages to you while you were lying in the stillness of night. So go now, and gladly prepare the young men to be armed and their march through the gates into the fields, and burn up the Phrygian leaders who have settled on our beautiful river and their painted ships too. The mighty power of the heavenly gods demands it. King Latinus himself, unless he agrees to obey his word and give you the marriage, let him feel it to his cost and at last face Turnus in arms.'
}

\end{document}
