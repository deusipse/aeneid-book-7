\documentclass[11pt, a4paper, landscape]{article}
\usepackage{fontspec}
\usepackage{geometry}
\geometry{
  top = 0.75in,
  bottom = 0.75in,
  left = 2in,
  right = 1.25in
}
\usepackage[series={}, nocritical, noeledsec, noend, nofamiliar, noledgroup]{reledmac}
\setlinenum{322}
\usepackage{reledpar}
\usepackage{microtype}
\usepackage{setspace}
% \usepackage{newtxtext}

\setmainfont{Times New Roman}
\setsansfont{Arial}

\usepackage{lipsum}
\usepackage{multicol}
\setlength\columnsep{1.75in}
\usepackage{parskip}
\usepackage[pdfusetitle, hidelinks]{hyperref}
\usepackage[english, main=latin]{babel}
\babeltags{english = english}

\newcommand{\Translation}[3]{
  \textbf{\large \sffamily #1}
  \begin{multicols}{2}
    \begin{spacing}{1.7}
      \beginnumbering
      \pstart
      #2
      \pend
      \pausenumbering
    \end{spacing}
    \columnbreak
    #3
  \end{multicols}
}
\newcommand{\translation}[3]{
  \textbf{\large \sffamily #1}
  \begin{multicols}{2}
    \begin{spacing}{1.7}
      \resumenumbering
      \pstart
      #2
      \pend
      \pausenumbering
    \end{spacing}
    \columnbreak
    #3
  \end{multicols}
}


\begin{document}

\Translation{323--340: Juno Summons Allecto}{
  haec ubi dicta dedit, terras horrenda petivit;\\
  luctificam Allecto dirarum ab sede dearum\\
  infernisque ciet tenebris, cui tristia bella\\
  iraeque insidiaeque et crimina noxia cordi.\\
  odit et ipse pater Pluton, odere sorores\\
  Tartareae monstrum: tot sese vertit in ora,\\
  tam saevae facies, tot pullulat atra colubris.\\
  quam Iuno his acuit verbis ac talia fatur:\\
  `hunc mihi da proprium, virgo sata Nocte, laborem,\\
  hanc operam, ne noster honos infractave cedat\\
  fama loco, neu conubiis ambire Latinum\\
  Aeneadae possint Italosve obsidere finis.\\
  tu potes unanimos armare in proelia fratres\\
  atque odiis versare domos, tu verbera tectis\\
  funereasque inferre faces, tibi nomina mille,\\
  mille nocendi artes. fecundum concute pectus,\\
  dissice compositam pacem, sere crimina belli;\\
  arma velit poscatque simul rapiatque iuventus.'
}{
  When she had spoken these words, the dreadful one sought the earth; she summoned grief bringing Allecto from the seat of the fearful Furies in the infernal darkness, dear to her heart are gloomy wars, anger, trickery, and injurious crimes. And her own father Pluto himself hates her, her sisters of Tartarus hate the monster: she changes herself into so many appearances, such savage forms, she sprouts so many black snakes. Juno stirred her up with these words and spoke such things:

  `Do me this service, virgin daughter of the Night, this task, so that our honour and reputation are not weakened and give way, and see to it that people of Aeneas are unable to win over Latinus with marriage and besiege the Italian lands. You can take brothers who love each other and make them take up arms against themselves in battle, and you can upturn homes with hatred and fill them with whips and funeral torches, you have a thousand names, a thousand ways of causing hurt. Shake your teeming chest, shatter the peace they have put together, sow the accusations of war; let the youth long for weapons, demand them, seize them, now!'
}
\newpage

\translation{341--358: Allecto Attacks Amata}{
  exim Gorgoneis Allecto infecta venenis\\
  principio Latium et Laurentis tecta tyranni\\
  celsa petit, tacitumque obsedit limen Amatae,\\
  quam super adventu Teucrum Turnique hymenaeis\\
  femineae ardentem curaeque iraeque coquebant.\\
  huic dea caeruleis unum de crinibus anguem\\
  conicit, inque sinum praecordia ad intima subdit,\\
  quo furibunda domum monstro permisceat omnem.\\
  ille inter vestis et levia pectora lapsus\\
  volvitur attactu nullo, fallitque furentem\\
  vipeream inspirans animam; fit tortile collo\\
  aurum ingens coluber, fit longae taenia vittae\\
  innectitque comas et membris lubricus errat.\\
  ac dum prima lues udo sublapsa veneno\\
  pertemptat sensus atque ossibus implicat ignem\\
  necdum animus toto percepit pectore flammam,\\
  mollius et solito matrum de more locuta est,\\
  multa super natae lacrimans Phrygiisque hymenaeis:
}{
  At once Allecto, steeped in the poisons of the Gorgons, first sought Latium and the lofty roofs of the Laurentine king, and took up position at the quiet threshold of Amata, who was being stirred up by womanly concerns and anger and was seething at the arrival of Trojans and at the wedding of Turnus. The goddess hurled one of the snakes from her dark blue hair at her, and plunged it into her breast deep into her heart, so that maddened by the monster, she would throw into disarray the whole house. It slid between her robes and smooth breasts, it coiled without her any notice, and without her knowing it, breathed its viperous breath into the frenzied one; the giant snake became a necklace of twisted gold around her neck, it became the hanging end of a long ribbon and fastened itself into her hair, slithering along her body. And while the first infection from the clammy poison was agitating her senses and entwining fire in her bones, before her mind perceived the flame in her whole chest, she spoke softly and in the usual manner of mothers, weeping greatly over the marriage of her daughter to the Phrygians.
}
\newpage

\translation{359--377: Amata Laments Lavinia's Marriage}{
  `exsulibusne datur ducenda Lavinia Teucris,\\
  o genitor, nec te miseret nataeque tuique?\\
  nec matris miseret, quam primo Aquilone relinquet\\
  perfidus alta petens abducta virgine praedo?\\
  at non sic Phrygius penetrat Lacedaemona pastor,\\
  Ledaeamque Helenam Troianas vexit ad urbes?\\
  quid tua sancta fides? quid cura antiqua tuorum\\
  et consanguineo totiens data dextera Turno?\\
  si gener externa petitur de gente Latinis,\\
  idque sedet, Faunique premunt te iussa parentis,\\
  omnem equidem sceptris terram quae libera nostris\\
  dissidet externam reor et sic dicere divos.\\
  et Turno, si prima domus repetatur origo,\\
  Inachus Acrisiusque patres mediaeque Mycenae.'\\
  his ubi nequiquam dictis experta Latinum\\
  contra stare videt, penitusque in viscera lapsum\\
  serpentis furiale malum totamque pererrat,\\
  tum vero infelix ingentibus excita monstris\\
  immensam sine more furit lymphata per urbem.
}{\phantom{}}

\end{document}
