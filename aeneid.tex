\documentclass[11pt, a4paper, landscape]{article}
\usepackage{fontspec}
\usepackage{geometry}
\geometry{
  top = 0.75in,
  bottom = 0.75in,
  left = 2in,
  right = 1.25in
}
\usepackage[series={}, nocritical, noeledsec, noend, nofamiliar, noledgroup]{reledmac}
\setlinenum{322}
\usepackage{reledpar}
\usepackage{microtype}
\usepackage{setspace}
% \usepackage{newtxtext}

\setmainfont{Times New Roman}
\setsansfont{Arial}

\usepackage{lipsum}
\usepackage{multicol}
\setlength\columnsep{1.75in}
\usepackage{parskip}
\usepackage[pdfusetitle, hidelinks]{hyperref}
\usepackage[english, main=latin]{babel}
\babeltags{english = english}

\newcommand{\Translation}[3]{
  \textbf{\large \sffamily #1}
  \begin{multicols}{2}
    \begin{spacing}{1.7}
      \beginnumbering
      \pstart
      #2
      \pend
      \pausenumbering
    \end{spacing}
    \columnbreak
    #3
  \end{multicols}
}
\newcommand{\translation}[3]{
  \textbf{\large \sffamily #1}
  \begin{multicols}{2}
    \begin{spacing}{1.7}
      \resumenumbering
      \pstart
      #2
      \pend
      \pausenumbering
    \end{spacing}
    \columnbreak
    #3
  \end{multicols}
}


\begin{document}

\Translation{323--340: Juno Summons Allecto}{
  haec ubi dicta dedit, terras horrenda petivit;\\
  luctificam Allecto dirarum ab sede dearum\\
  infernisque ciet tenebris, cui tristia bella\\
  iraeque insidiaeque et crimina noxia cordi.\\
  odit et ipse pater Pluton, odere sorores\\
  Tartareae monstrum: tot sese vertit in ora,\\
  tam saevae facies, tot pullulat atra colubris.\\
  quam Iuno his acuit verbis ac talia fatur:\\
  `hunc mihi da proprium, virgo sata Nocte, laborem,\\
  hanc operam, ne noster honos infractave cedat\\
  fama loco, neu conubiis ambire Latinum\\
  Aeneadae possint Italosve obsidere finis.\\
  tu potes unanimos armare in proelia fratres\\
  atque odiis versare domos, tu verbera tectis\\
  funereasque inferre faces, tibi nomina mille,\\
  mille nocendi artes. fecundum concute pectus,\\
  dissice compositam pacem, sere crimina belli;\\
  arma velit poscatque simul rapiatque iuventus.'
}{
  When she had spoken these words, the dreadful one sought the earth; she summoned grief bringing Allecto from the seat of the fearful Furies in the infernal darkness, dear to her heart are gloomy wars, anger, trickery, and injurious crimes. And her own father Pluto himself hates her, her sisters of Tartarus hate the monster: she changes herself into so many appearances, such savage forms, she sprouts so many black snakes. Juno stirred her up with these words and spoke such things:

  `Do me this service, virgin daughter of the Night, this task, so that our honour and reputation are not weakened and give way, and see to it that people of Aeneas are unable to win over Latinus with marriage and besiege the Italian lands. You can take brothers who love each other and make them take up arms against themselves in battle, and you can upturn homes with hatred and fill them with whips and funeral torches, you have a thousand names, a thousand ways of causing hurt. Shake your teeming chest, shatter the peace they have put together, sow the accusations of war; let the youth long for weapons, demand them, seize them, now!'
}
\newpage

\translation{341--358: Allecto Attacks Amata}{
  exim Gorgoneis Allecto infecta venenis\\
  principio Latium et Laurentis tecta tyranni\\
  celsa petit, tacitumque obsedit limen Amatae,\\
  quam super adventu Teucrum Turnique hymenaeis\\
  femineae ardentem curaeque iraeque coquebant.\\
  huic dea caeruleis unum de crinibus anguem\\
  conicit, inque sinum praecordia ad intima subdit,\\
  quo furibunda domum monstro permisceat omnem.\\
  ille inter vestis et levia pectora lapsus\\
  volvitur attactu nullo, fallitque furentem\\
  vipeream inspirans animam; fit tortile collo\\
  aurum ingens coluber, fit longae taenia vittae\\
  innectitque comas et membris lubricus errat.\\
  ac dum prima lues udo sublapsa veneno\\
  pertemptat sensus atque ossibus implicat ignem\\
  necdum animus toto percepit pectore flammam,\\
  mollius et solito matrum de more locuta est,\\
  multa super natae lacrimans Phrygiisque hymenaeis:
}{
  At once Allecto, steeped in the poisons of the Gorgons, first sought Latium and the lofty roofs of the Laurentine king, and took up position at the quiet threshold of Amata, who was being stirred up by womanly concerns and anger and was seething at the arrival of Trojans and at the wedding of Turnus. The goddess hurled one of the snakes from her dark blue hair at her, and plunged it into her breast deep into her heart, so that maddened by the monster, she would throw into disarray the whole house. It slid between her robes and smooth breasts, it coiled without her any notice, and without her knowing it, breathed its viperous breath into the frenzied one; the giant snake became a necklace of twisted gold around her neck, it became the hanging end of a long ribbon and fastened itself into her hair, slithering along her body. And while the first infection from the clammy poison was agitating her senses and entwining fire in her bones, before her mind perceived the flame in her whole chest, she spoke softly and in the usual manner of mothers, weeping greatly over the marriage of her daughter to the Phrygians.
}
\newpage

\translation{359--377: Amata Laments Lavinia's Marriage}{
  `exsulibusne datur ducenda Lavinia Teucris,\\
  o genitor, nec te miseret nataeque tuique?\\
  nec matris miseret, quam primo Aquilone relinquet\\
  perfidus alta petens abducta virgine praedo?\\
  at non sic Phrygius penetrat Lacedaemona pastor,\\
  Ledaeamque Helenam Troianas vexit ad urbes?\\
  quid tua sancta fides? quid cura antiqua tuorum\\
  et consanguineo totiens data dextera Turno?\\
  si gener externa petitur de gente Latinis,\\
  idque sedet, Faunique premunt te iussa parentis,\\
  omnem equidem sceptris terram quae libera nostris\\
  dissidet externam reor et sic dicere divos.\\
  et Turno, si prima domus repetatur origo,\\
  Inachus Acrisiusque patres mediaeque Mycenae.'\\
  his ubi nequiquam dictis experta Latinum\\
  contra stare videt, penitusque in viscera lapsum\\
  serpentis furiale malum totamque pererrat,\\
  tum vero infelix ingentibus excita monstris\\
  immensam sine more furit lymphata per urbem.
}{
  `Must Lavinia be led and given in marriage to Trojan exiles, O father, do you have no pity for your daughter or for yourself? Nor do you pity her mother, whom the treacherous pirate will abandon, seeking the high seas as soon as the North wind blows, taking away our virgin? But is this not how the Phrygian shepherd made his way into Lacedaemon, and dragged Helen, the daughter of Leda, off to the cities of Troy? What of your sacred pledge? What of your ancient care for your people, and your right hand so often given to your kinsman Turnus? If a son-in-law is being sought from a people foreign to the Latins, and that is decided upon, if the commands of your father Faunus weigh upon you, then I personally think that all land that is free and separate from our rule is foreign, and such is what the gods say. And if the first origin of the house of Turnus was traced back, his fathers were Inachus and Acrisius and the middle of Mycenae.'

  When having tried in vain with these words she saw that Latinus was standing against her, and the snake's maddening venom had soaked into her flesh and was coursing through her whole body, then indeed the unlucky queen, driven by the monstrous horrors, raged frantically from end to end of the city without regard for convention.
}
\newpage

\translation{378--396: Amata Rages Through the City}{
  ceu quondam torto volitans sub verbere turbo,\\
  quem pueri magno in gyro vacua atria circum\\
  intenti ludo exercent---ille actus habena\\
  curvatis fertur spatiis; stupet inscia supra\\
  impubesque manus mirata volubile buxum;\\
  dant animos plagae: non cursu segnior illo\\
  per medias urbes agitur populosque ferocis.\\
  quin etiam in silvas simulato numine Bacchi\\
  maius adorta nefas maioremque orsa furorem\\
  evolat et natam frondosis montibus abdit,\\
  quo thalamum eripiat Teucris taedasque moretur,\\
  euhoe Bacche fremens, solum te virgine dignum\\
  vociferans: etenim mollis tibi sumere thyrsos,\\
  te lustrare choro, sacrum tibi pascere crinem.\\
  fama volat, furiisque accensas pectore matres\\
  idem omnis simul ardor agit nova quaerere tecta.\\
  deseruere domos, ventis dant colla comasque;\\
  ast aliae tremulis ululatibus aethera complent\\
  pampineasque gerunt incinctae pellibus hastas.
}{
  Just like sometimes a spinning top, flying under the whirled whip, which boys, engrossed in their game, make go in a great circle around an empty court---driven on by the whip it speeds in circular courses; the childish group hang over it in wonder, mesmerised by what they have never seen before, gazing at the twirling boxwood; their strokes give it life: no slower than the spinning top she is driven through the midst of the cities and proud peoples. And moreover she flew into the forests feigning the spirit of Bacchus, she rose to a greater evil and greater madness, and she hid her daughter in the leafy mountains hoping to snatch the marriage away from the Trojans and delay the marriage torches, `euhoe, Bacchus!' she screams, `you alone are worthy of the virgin! For you in truth she lifts the soft thyrsus and moves around you in ritual dance, she grows her sacred hair for you.' Rumour flew, and the same flame in their breasts drove all the mothers, inflamed with frenzy, to seek new homes. They abandoned their houses, baring their neck and hair to the winds; some filled the heavens with tremulous wailing and clad in animal skins, took up vine-draped spears.
}
\newpage

\translation{397--414: Allecto Seeks Out Turnus}{
  ipsa inter medias flagrantem fervida pinum\\
  sustinet ac natae Turnique canit hymenaeos\\
  sanguineam torquens aciem, torvumque repente\\
  clamat: `io matres, audite, ubi quaeque, Latinae:\\
  si qua piis animis manet infelicis Amatae\\
  gratia, si iuris materni cura remordet,\\
  solvite crinalis vittas, capite orgia mecum.'\\
  talem inter silvas, inter deserta ferarum\\
  reginam Allecto stimulis agit undique Bacchi.\\
  postquam visa satis primos acuisse furores\\
  consiliumque omnemque domum vertisse Latini,\\
  protinus hinc fuscis tristis dea tollitur alis\\
  audacis Rutuli ad muros, quam dicitur urbem\\
  Acrisioneis Danae fundasse colonis\\
  praecipiti delata Noto. locus Ardea quondam\\
  dictus avis, et nunc magnum manet Ardea nomen,\\
  sed fortuna fuit. tectis hic Turnus in altis\\
  iam mediam nigra carpebat nocte quietem.
}{
  The fiery queen herself held up a burning pine and sang wedding hymns for Turnus and her daughter, rolling her bloodshot eyes, suddenly she gave a wild shout: 

  `Io, mothers of Latium, listen, wherever you are: if any regard for unlucky Amata remains in your faithful hearts, if concern for a mother's rights bites at you, loosen your headbands and join me in the secret rites!'

  Such was the queen whom Allecto drove far and wide through the woods and the deserted lairs of wild beasts with the lash of Bacchus. After she saw that she had stirred the first frenzies enough and had subverted the plans and the whole household of Latinus, the gloomy goddess was carried on her dark wings straight to the walls of the bold Rutilian, the city which Danae is said to have founded for the Acrisionean settlers, having been carried there by the swift South wind. The place was once called Ardea by our ancestors, and now the great name Ardea remains, but its greatness is gone. Here in his high palace in the dark of night, Turnus was lying in the middle of his sleep.
}
\newpage

\translation{415--434: Allecto Rouses Turnus}{
  Allecto torvam faciem et furialia membra\\
  exuit, in vultus sese transformat anilis\\
  et frontem obscenam rugis arat, induit albos\\
  cum vitta crinis, tum ramum innectit olivae;\\
  fit Calybe Iunonis anus templique sacerdos,\\
  et iuveni ante oculos his se cum vocibus offert:\\
  `Turne, tot incassum fusos patiere labores,\\
  et tua Dardaniis transcribi sceptra colonis?\\
  rex tibi coniugium et quaesitas sanguine dotes\\
  abnegat, externusque in regnum quaeritur heres.\\
  i nunc, ingratis offer te, inrise, periclis;\\
  Tyrrhenas, i, sterne acies, tege pace Latinos.\\
  haec adeo tibi me, placida cum nocte iaceres,\\
  ipsa palam fari omnipotens Saturnia iussit.\\
  quare age et armari pubem portisque moveri\\
  laetus in arva para, et Phrygios qui flumine pulchro\\
  consedere duces pictasque exure carinas.\\
  caelestum vis magna iubet. rex ipse Latinus,\\
  ni dare coniugium et dicto parere fatetur,\\
  sentiat et tandem Turnum experiatur in armis.'
}{
  Allecto shook off her savage appearance and her fury's writhing limbs, she transformed herself into the form of an old woman and furrowed her brow with repulsive wrinkles, she put on white hair with a fillet, then fastened an olive wreath there; she became Calybe, the aged priestess of Juno, and before the eyes of the young man she offered herself with these words:

  `Turnus, will you allow so many efforts of yours to be wasted in vain, and your sceptre handed over to Dardanian colonists? The king denies you your marriage and dowry earned in blood, and a stranger is being sought as heir to the kingdom. Go now, offer yourself, laughed at, to those unthankful dangers; go and cut down the Tyrrhenian battle lines, protect the Latins with peace! The all-powerful Saturnian goddess herself commanded me to openly give these messages to you while you were lying in the stillness of night. So go now, and gladly prepare the young men to be armed and their march through the gates into the fields, and burn up the Phrygian leaders who have settled on our beautiful river and their painted ships too. The mighty power of the heavenly gods demands it. King Latinus himself, unless he agrees to obey his word and give you the marriage, let him feel it to his cost and at last face Turnus in arms.'
}
\newpage

\translation{435--455: Allecto Responds to Turnus’ Mocks}{
  Hic iuvenis vatem inridens sic orsa vicissim\\
  ore refert: `classis invectas Thybridis undam\\
  non, ut rere, meas effugit nuntius auris;\\
  ne tantos mihi finge metus. nec regia Iuno\\
  immemor est nostri.\\
  sed te victa situ verique effeta senectus,\\
  o mater, curis nequiquam exercet, et arma\\
  regum inter falsa vatem formidine ludit.\\
  cura tibi divum effigies et templa tueri;\\
  bella viri pacemque gerent quis bella gerenda.'\\
  talibus Allecto dictis exarsit in iras.\\
  at iuveni oranti subitus tremor occupat artus,\\
  deriguere oculi: tot Erinys sibilat hydris\\
  tantaque se facies aperit; tum flammea torquens\\
  lumina cunctantem et quaerentem dicere plura\\
  reppulit, et geminos erexit crinibus anguis,\\
  verberaque insonuit rabidoque haec addidit ore:\\
  `en ego victa situ, quam veri effeta senectus\\
  arma inter regum falsa formidine ludit.\\
  respice ad haec: adsum dirarum ab sede sororum,\\
  bella manu letumque gero.'
}{
  Thereupon the young man, laughing at the prophet in this way opened his mouth in reply:

`News that a fleet has sailed into the waters of the Tiber has not, as you think, escaped my ears; do not invent such fears for me. Nor is queen Juno unmindful of us. But old age needlessly harasses you O mother, overcome by decay and devoid of truth, with worries, and makes a fool of you, though a prophet, with false fears amidst warring kings. Your concern is to look after the statues and temples of the gods; let the men manage war and peace, whose work is war.'

At such words Allecto blazed into anger. But a sudden tremor seized his limbs as he spoke, his eyes became fixed: the Fury hissed with so many snakes and such a form revealed itself; then, rolling her flaming eyes, she thrust him back faltering and trying to say more, and raised a pair of snakes from her hair, she cracked her whips and added these things from her raving mouth:

`So here I am, overcome by decay and devoid of truth, whom old age mocks with false fears amidst warring kings. Look at these things! Here I am from the seat of the fearful sisters, I bring wars and destruction with my hand.'
}
\newpage

\translation{456--474: Turnus is Unleashed}{
  sic effata facem iuveni coniecit et atro\\
  lumine fumantis fixit sub pectore taedas.\\
  olli somnum ingens rumpit pavor, ossaque et artus\\
  perfundit toto proruptus corpore sudor.\\
  arma amens fremit, arma toro tectisque requirit;\\
  saevit amor ferri et scelerata insania belli,\\
  ira super: magno veluti cum flamma sonore\\
  virgea suggeritur costis undantis aeni\\
  exsultantque aestu latices, furit intus aquai\\
  fumidus atque alte spumis exuberat amnis,\\
  nec iam se capit unda, volat vapor ater ad auras.\\
  ergo iter ad regem polluta pace Latinum\\
  indicit primis iuvenum et iubet arma parari,\\
  tutari Italiam, detrudere finibus hostem;\\
  se satis ambobus Teucrisque venire Latinisque.\\
  haec ubi dicta dedit divosque in vota vocavit,\\
  certatim sese Rutuli exhortantur in arma.\\
  hunc decus egregium formae movet atque iuventae,\\
  hunc atavi reges, hunc claris dextera factis.
}{
  Thus having spoken, she hurled a torch at the young man and drove it, smoking with black light, into the depths of his chest. An immense terror burst his sleep, and sweat burst out from his whole body and poured over his bones and limbs. Frenzied, he roared for his armour, and searched for weapons in his bed and house; the lust for the sword raged, the criminal madness of war, and anger above all: just as when blazing sticks are piled up with a loud crackling under the ribs of a billowing cauldron and water leaps about with the heat, within rages the steaming stream of water and it bubbles high with foam, the wave contains itself no longer and black smoke soars aloft into the air. Therefore Turnus commanded the leaders of his men to march on King Latinus with the peace violated, and he ordered arms to be prepared, Italy to be defended, and to drive out the enemy from the borders; he said that his coming would be a match for both the Trojans and the Latins. When he had given these words and called the gods to witness his vows, the Rutulians eagerly urged each other to arms. This man is moved by the outstanding grace of his form and youth, that one by his royal ancestry, and yet another by the glorious deeds of his right hand.
}
\newpage

\translation{475--492: Allecto Seeks the Stag}{
  Dum Turnus Rutulos animis audacibus implet, \\
  Allecto in Teucros Stygiis se concitat alis, \\
  arte nova, speculata locum, quo litore pulcher \\
  insidiis cursuque feras agitabat Iulus. \\
  hic subitam canibus rabiem Cocytia virgo \\
  obicit et noto naris contingit odore, \\
  ut cervum ardentes agerent; quae prima laborum \\
  causa fuit belloque animos accendit agrestis. \\
  cervus erat forma praestanti et cornibus ingens, \\
  Tyrrhidae pueri quem matris ab ubere raptum \\
  nutribant Tyrrhusque pater, cui regia parent \\
  armenta et late custodia credita campi. \\
  adsuetum imperiis soror omni Silvia cura \\
  mollibus intexens ornabat cornua sertis, \\
  pectebatque ferum puroque in fonte lauabat. \\
  ille manum patiens mensaeque adsuetus erili \\
  errabat silvis rursusque ad limina nota \\
  ipse domum sera quamvis se nocte ferebat.
}{
  While Turnus was filling the Rutulians with daring courage, Allecto raced off with her Stygian wings to the Trojans, with a new technique, spying a place on a shore where handsome Iulus was hunting wild beasts with nets on horse back. Here the Cocytian virgin hurled a sudden madness at the dogs, and infects their nostrils with a well known scent, so that the hounds would ardently pursue a stag; this was the first cause of the troubles and inflamed the minds of the countrymen with war. There was a huge stag of outstanding shape and with antlers, who was snatched from the breast of its mother, the sons of Tyrrhus nurtured it, along with Tyrrhus their father, whom the royal herds obey, and to whom guardianships of the pastor had been entrusted far and wide. It was accustomed to the commands of their sister Silvia, and she used to adorn its horns with every care, weaving soft garlands around them. And she would comb the wild beast and bathe it in a fresh spring. That beast, enduring her hand, and accustomed to food from the master's table, was wandering in the forests and however late at night it would take itself back home again to the well known thresholds.
}
\newpage

\translation{493--510: Ascanius Shoots the Stag}{
  hunc procul errantem rabidae venantis Iuli \\
  commovere canes, fluvio cum forte secundo \\
  deflueret ripaque aestus viridante levaret. \\
  ipse etiam eximiae laudis succensus amore \\
  Ascanius curvo derexit spicula cornu; \\
  nec dextrae erranti deus afuit, actaque multo \\
  perque uterum sonitu perque ilia venit harundo. \\
  saucius at quadripes nota intra tecta refugit \\
  successitque gemens stabulis, questuque cruentus \\
  atque imploranti similis tectum omne replebat. \\
  Silvia prima soror palmis percussa lacertos \\
  auxilium vocat et duros conclamat agrestis. \\
  olli (pestis enim tacitis latet aspera silvis) \\
  improvisi adsunt, hic torre armatus obusto, \\
  stipitis hic gravidi nodis; quod cuique repertum \\
  rimanti telum ira facit. vocat agmina Tyrrhus, \\
  quadrifidam quercum cuneis ut forte coactis \\
  scindebat rapta spirans immane securi.
}{
  The rabid dogs stirred up this beast wandering far from the hunting Iulus, when by chance it floated downstream and cooled off the heat on a grassy bank. Ascanius himself, enflamed with love of extraordinary glory, aimed arrows from his bow; nor was the god absent from his wandering right hand and the arrow fired with lots of noise went through its belly and guts. But the wounded quadruped fled back into its well known abode and crawled into its stable, groaning and bloodied, it filled the whole building with its complaints like a person begging for help. At first sister Silvia, beating her arms with her hands, calls for help and cries out to the hardy country-folk. They arrive unforeseen (for the harsh fiend lurks hidden in the silent woods), one is armed with a burnt firebrand, another with a heavy knotted club; anger makes a weapon of what has been found by each person in his search. Tyrrhus called the battle lines, since by chance he was splitting an oak into four by driving in wedges, breathing heavily, he seized his axe.
}
\newpage

\translation{511--530: Battle Ensues}{
  At saeva e speculis tempus dea nacta nocendi \\
  ardua tecta petit stabuli et de culmine summo \\
  pastorale canit signum cornuque recurvo \\
  Tartaream intendit vocem, qua protinus omne \\
  contremuit nemus et silvae insonuere profundae; \\
  audiit et Triviae longe lacus, audiit amnis \\
  sulpurea Nar albus aqua fontesque Velini, \\
  et trepidae matres pressere ad pectora natos. \\
  tum vero ad vocem celeres, qua bucina signum \\
  dira dedit, raptis concurrunt undique telis \\
  indomiti agricolae, nec non et Troia pubes \\
  Ascanio auxilium castris effundit apertis. \\
  derexere acies. non iam certamine agresti \\
  stipitibus duris agitur sudibusve praeustis, \\
  sed ferro ancipiti decernunt atraque late \\
  horrescit strictis seges ensibus, aeraque fulgent \\
  sole lacessita et lucem sub nubila iactant: \\
  fluctus uti primo coepit cum albescere vento, \\
  paulatim sese tollit mare et altius undas \\
  erigit, inde imo consurgit ad aethera fundo.
}{
But the savage goddess, having obtained a time to do harm from her watchtower, sought the lofty roofs of the stable and sounded the shepherd's signal down from the highest peak and strained the Tartarean sound with a recurved horn, with which she suddenly shook the whole grove and the deep forests resounded; even the distant lake of Diana heard, the river Nar, white with its sulphurous water, and the springs of Velinus heard, and restless mothers pressed their children to their breasts. Then in truth the untamed farmers, quick to the sound, with which the dire trumpet gave the signal, ran together when weapons had been seized from all sides, and the Trojan youth also poured out when the camps had been opened up, as help for Ascanius. They deployed their battle lines. They do not fight now with heavy clubs or fire-hardened stakes, like in some country brawl, but they fought with double edged swords and a black crop bristles far and wide with drawn swords, and bronze armour, struck by the sun, shone, and threw light up to the clouds: as when a wave begins to grow white at the first wind and the sea swells gradually and lifts up its waves higher, it surges up to the sky from the lowest depth.
}
\newpage

\translation{531--551: Bloodshed on the Battlefield}{
  hic iuvenis primam ante aciem stridente sagitta, \\
  natorum Tyrrhi fuerat qui maximus, Almo, \\
  sternitur; haesit enim sub gutture vulnus et udae \\
  vocis iter tenuemque inclusit sanguine vitam. \\
  corpora multa virum circa seniorque Galaesus, \\
  dum paci medium se offert, iustissimus unus \\
  qui fuit Ausoniisque olim ditissimus arvis: \\
  quinque greges illi balantum, quina redibant \\
  armenta, et terram centum vertebat aratris. \\
  Atque ea per campos aequo dum Marte geruntur, \\
  promissi dea facta potens, ubi sanguine bellum \\
  imbuit et primae commisit funera pugnae, \\
  deserit Hesperiam et caeli conversa per auras \\
  Iunonem victrix adfatur voce superba: \\
  `en, perfecta tibi bello discordia tristi; \\
  dic in amicitiam coeant et foedera iungant. \\
  quandoquidem Ausonio respersi sanguine Teucros, \\
  hoc etiam his addam, tua si mihi certa voluntas: \\
  finitimas in bella feram rumoribus urbes, \\
  accendamque animos insani Martis amore \\
  undique ut auxilio veniant; spargam arma per agros.'
}{
  At this point, the young man Almo, who had been the oldest of the sons of Tyrrhus, was struck down by a hissing arrow standing in the front rank; for the wound stuck under his throat and choked off the passage of his damp voice and his thin breath with blood. Many bodies of men fell around him, and among them was the old man Galaesus, killed while he was offering himself for peace in the middle, he who was singularly the most righteous man and once the richest in the Italian fields: for him five flocks of sheep bleated, five herds of cattle came back to him, and he was turning the soil with a hundred ploughs.

  And while those things were being waged through the plains with equal Mars, the goddess having fulfilled her promise, when the battle had been stained with blood and the first battle had been joined with blood, she abandoned Italy and, having turned through the airs of the sky, she addressed Juno as the victor, with a proud voice: 
  `See, discord has been brought to you with gloomy wars; tell them to get together in friendship and join in treaty. Seeing that I have sprinkled the Trojans with Ausonian blood, I will also add this to these things, if I understand your intentions right: I will bring the neighbouring cities into war with rumours, and I will enflame their minds with mad love for Mars on all sides in order to bring reinforcement; I will sprinkle arms on the fields.'
}
\newpage

\translation{552--571: Allecto is Dismissed}{
  tum contra Iuno: `terrorum et fraudis abunde est: \\
  stant belli causae, pugnatur comminus armis, \\
  quae fors prima dedit sanguis novus imbuit arma. \\
  talia coniugia et talis celebrent hymenaeos \\
  egregium Veneris genus et rex ipse Latinus. \\
  te super aetherias errare licentius auras \\
  haud pater ille velit, summi regnator Olympi. \\
  cede locis. ego, si qua super fortuna laborum est, \\
  ipsa regam.' talis dederat Saturnia voces; \\
  illa autem attollit stridentis anguibus alas \\
  Cocytique petit sedem supera ardua linquens. \\
  est locus Italiae medio sub montibus altis, \\
  nobilis et fama multis memoratus in oris, \\
  Amsancti valles; densis hunc frondibus atrum \\
  urget utrimque latus nemoris, medioque fragosus \\
  dat sonitum saxis et torto vertice torrens. \\
  hic specus horrendum et saevi spiracula Ditis \\
  monstrantur, ruptoque ingens Acheronte vorago \\
  pestiferas aperit fauces, quis condita Erinys, \\
  invisum numen, terras caelumque levabat.
}{
  Then Juno refuted: 
  `There is more than enough terror and deceit: the causes of war stand, they fight hand to hand, new blood is staining weapons, which chance first provided; let them celebrate such marriages and such weddings, let the excellent offspring of Venus and King Latinus himself celebrate. That father, the ruler of high Olympus would not want you to wander freely in the ethereal heavens. Withdraw from this place. I myself will handle it, if any further troubles should chance to occur.' The Saturnian one had spoken such words; Allecto lifts her wings hissing with snakes and sought her seat in Cocytus, leaving the upper air. There is a place in the middle of Italy at the foot of tall mountains, renowned and mentioned by tradition in many regions, the Valley of Amsanctus; a fringe of woodland hems it in on both sides, dark with thick foliage, and a roaring torrent resounds in the rock and twisting whirlpool. Here a frightening cave and a breathing hole of cruel Dis are shown and a huge chasm opened up its destructive jaws where Acheron has burst through, in which the fury buried itself, a hated goddess, and relieved the lands and sky.
}
\newpage

\translation{572--590: People Call for War}{
  Nec minus interea extremam Saturnia bello \\
  imponit regina manum. ruit omnis in urbem \\
  pastorum ex acie numerus, caesosque reportant \\
  Almonem puerum foedatique ora Galaesi, \\
  implorantque deos obtestanturque Latinum. \\
  Turnus adest medioque in crimine caedis et igni \\
  terrorem ingeminat: Teucros in regna vocari, \\
  stirpem admisceri Phrygiam, se limine pelli. \\
  tum quorum attonitae Baccho nemora avia matres \\
  insultant thiasis (neque enim leve nomen Amatae) \\
  undique collecti coeunt Martemque fatigant. \\
  ilicet infandum cuncti contra omina bellum, \\
  contra fata deum perverso numine poscunt. \\
  certatim regis circumstant tecta Latini; \\
  ille velut pelago rupes immota resistit, \\
  ut pelagi rupes magno veniente fragore, \\
  quae sese multis circum latrantibus undis \\
  mole tenet; scopuli nequiquam et spumea circum \\
  saxa fremunt laterique inlisa refunditur alga.
}{
  Meanwhile the Saturnian queen puts a finishing touch on the war. The whole number of shepherds runs into the city out of the battle line, and carried back the dead, the boy Almo, the mangled face of Galaesus, and begged the gods for help and called Latinus to witness. Turnus arrives, and in the midst of the criminal slaughter and fire, he doubles the terror: he said that the Trojans were being called into the kingdom, that Trojan descendants were being mixed in, and that he was being driven from the threshold. Then they, whose mothers, bewitched by Bacchus, prance around the pathless woodlands in dancing tropes (for the name of Amata is not insignificant), drawing together from all sides, gathered and wore out Mars. Immediately, with perverse will, all of them demanded unholy war, despite the omens, despite the pronouncements of the gods. They eagerly surround the palace of King Latinus; he resists them like an unmoved cliff resists the sea, like a cliff facing the open sea when a great crashing comes, which holds its ground due to its mass while many waves bark around; the rock stacks and the foam covered rocks roar around it in vain and seaweed has been hurled against the sides and washed back.
}

\newpage

\translation{591--600: Latinus Withdraws}{
  verum ubi nulla datur caecum exsuperare potestas \\
  consilium, et saevae nutu Iunonis eunt res, \\
  multa deos aurasque pater testatus inanis \\
  `frangimur heu fatis' inquit `ferimurque procella! \\
  ipsi has sacrilego pendetis sanguine poenas, \\
  o miseri. te, Turne, nefas, te triste manebit \\
  supplicium, votisque deos venerabere seris. \\
  nam mihi parta quies, omnisque in limine portus \\
  funere felici spolior.' nec plura locutus \\
  saepsit se tectis rerumque reliquit habenas.
}{
  But when he has no power to overcome their blind resolve and when things are going in accordance with the nod of cruel Juno, the father, with many appeals to the gods and the empty airs: `we are being broken by fate', he said, `we are struck by storms! You yourselves will pay the penalty with your sacrilegious blood O wretched ones. A crime awaits you, Turnus, and a dire punishment, and you will pray to the gods with prayers that are all too late. For rest has been produced for me, and on the threshold of every harbour I am stripped of a happy death.' Saying no more he enclosed himself in his palace and abandoned the reins of power.
}

\newpage

\translation{601--619: The Gates of War}{
   Mos erat Hesperio in Latio, quem protinus urbes
Albanae coluere sacrum, nunc maxima rerum
Roma colit, cum prima movent in proelia Martem,
sive Getis inferre manu lacrimabile bellum
Hyrcanisve Arabisve parant, seu tendere ad Indos
Auroramque sequi Parthosque reposcere signa:
sunt geminae Belli portae (sic nomine dicunt)
religione sacrae et saevi formidine Martis;
centum aerei claudunt vectes aeternaque ferri
robora, nec custos absistit limine Ianus.
has, ubi certa sedet patribus sententia pugnae,
ipse Quirinali trabea cinctuque Gabino
insignis reserat stridentia limina consul,
ipse vocat pugnas; sequitur tum cetera pubes,
aereaque adsensu conspirant cornua rauco.
hoc et tum Aeneadis indicere bella Latinus
more iubebatur tristisque recludere portas.
abstinuit tactu pater aversusque refugit
foeda ministeria, et caecis se condidit umbris.
}{
  There was a custom in Hesperian Latium, which from that time onwards, the Alban cities practised as a sacred thing, now great Rome also, when they are moving Mars into first battles, whether they are preparing to wage sorrowful war in force against the Getae or the Hycananians or the Arabs, or preparing to march on the Indians and follow Aurora and demand the standards from the Parthians: there are two twin gates of War (thus they call them), sacred by religion and dread for fierce Mars; a hundred bronze bolts close them and the eternal hardness of iron, and Janus the guard is never away from the threshold. The consul himself unlocks these creaking gates, distinguished in a Quirinal toga and Gabinian knotting, when the decision for war sits certain with the senators, he himself calls for war; then the rest of the young warriors follow suit, and bronze bugles sound with harsh assent. In keeping with this custom, Latinus was being ordered to declare war against the followers of Aeneas and open up the gloomy gates. The father refused to touch the fates and, having turned away, shunned the horrible duty, and buried himself in the blind shadows.
}

\newpage

\translation{620--640: War Officially Begins}{
  tum regina deum caelo delapsa morantis
  impulit ipsa manu portas, et cardine verso
  Belli ferratos rumpit Saturnia postis.
  ardet inexcita Ausonia atque immobilis ante;
  pars pedes ire parat campis, pars arduus altis
  pulverulentus equis furit; omnes arma requirunt.
  pars levis clipeos et spicula lucida tergent
  arvina pingui subiguntque in cote securis;
  signaque ferre iuvat sonitusque audire tubarum.
  quinque adeo magnae positis incudibus urbes
  tela novant, Atina potens Tiburque superbum,
  Ardea Crustumerique et turrigerae Antemnae.
  tegmina tuta cavant capitum flectuntque salignas
  umbonum cratis; alii thoracas aenos
  aut levis ocreas lento ducunt argento;
  vomeris huc et falcis honos, huc omnis aratri
  cessit amor; recoquunt patrios fornacibus ensis.
  classica iamque sonant, it bello tessera signum;
  hic galeam tectis trepidus rapit, ille trementis
  ad iuga cogit equos, clipeumque auroque trilicem
  loricam induitur fidoque accingitur ense.
}{
  Then the queen of the gods, sliding down from heaven, pushed the reluctant gates with her own hand, and the Saturnian one bursts open the iron-clad doors of War, turning them on their hinges. Italy, previously unstirred and unmoving, now burns; some prepared to go on the plain as infantry, others, elevated on high horses, stirred the deep dust; all of them call for their arms. Some are cleaning their smooth shields and bright javelins with rich fat and they sharpen their axes on the whetstone; some enjoyed carrying standards and listening to the sounds of the trumpet. In fact, five great cities set up anvils and forged new weapons, Powerful Atina and proud Tibur, Ardea, Crustumerium and towered Anthemnae. They forged helmets to protect their heads and weaved willow wickerworks of shields; some beat out bronze breast plates or polished greaves of pliant silver; pride in the ploughshore and the sickle and the entire love of the plough has given way to this; they reforge ancestral swords in furnaces. Trumpets now sound, the signal for war goes out; this man excitedly grabs his helmet from his house, another man drives his quivering horses to the yoke, and he puts on a shield and a breastplate triple woven with gold and he girds on his trusty sword.
}

\end{document}
